\chapter*{Abstract}

Electric energy generation is expect to undergo important changes at
the XXI century. The cost of wind energy electricity generation has
dropped more than 80\% lately due to the development of new conversion
technologies. Energy generation through WECS (wind energy conversion
systems) is expected to supply a considerable amount of electricity.
In this context, wind turbines equipped with permanent magnet
synchronous generators (PMSG) present relevant advantages over other
types of electric generators. However, PMSGs must be connected to
full-scale power electronic converters that provide a suitable
interface to the power grid. 
These converters allow a WECS to be operated at its maximum power at each time instant. Furthermore,
modern power converters guarantee high power quality, high conversion
efficiency levels, reliability and flexibility. 
This work proposes a novel high performance/power factor PMSG current control technique.
The proposed control technique is based on the current self-control
applied to three-phase rectifiers and improves it by including a
reactive power compensation for the PMSG. 
A three-phase PWM rectifier is employed to achieve robustness and low conduction losses. The
analysis of such a rectifier in a WECS is presented along with the
proposal of an appropriate space vector modulation scheme. 
The analyzed WECS is theoretically modeled and tested through computer
simulations and the proposed current control technique is
experimentally verified in a 6.5 kVA lab-prototype.

\vspace{5mm}

\begin{flushleft}

Keywords: current self-control, digital control, modeling, permanent magnet synchronous generator, PLL, space
vector modulation, three-phase PWM rectifier, wind energy
conversion system.

\end{flushleft}